\documentclass[12pt, leqno]{article}
\usepackage{amsfonts}
\usepackage{amsmath}
\usepackage{fancyhdr}
\usepackage{hyperref}
\usepackage{tikz}
\usepackage{pgfplots}
\usepackage{listings}

\newcommand{\bbR}{\mathbb{R}}
\newcommand{\bbC}{\mathbb{C}}

\newcommand{\hdr}[2]{
  \pagestyle{fancy}
  \lhead{Bindel, Spring 2015}
  \rhead{Numerical Analysis (CS 4220)}
  \fancyfoot{}
  \begin{center}
    {\large{\bf #1}} \\
    Due: #2
  \end{center}
  \lstset{language=matlab,columns=flexible}
}

\newcommand{\phdr}[1]{
  \pagestyle{fancy}
  \lhead{Bindel, Spring 2015}
  \rhead{Numerical Analysis (CS 4220)}
  \fancyfoot{}
  \begin{center}
    {\large{\bf #1}}
  \end{center}
  \lstset{language=matlab,columns=flexible}
}


\newcommand{\calK}{\mathcal{K}}
\newcommand{\calP}{\mathcal{P}}
\newcommand{\calR}{\mathcal{R}}

\begin{document}
\hdr{2015-03-11}

\section*{The Big Picture}

In this lecture, we start with our discussion of Krylov subspace
methods in general, and the famous method of conjugate gradients (CG)
in particular.  Though this is the iterative method of choice for most
positive definite systems, it may be as famously confusing as it is
famous\footnote{%
  See, e.g., ``Introduction to the Conjugate Gradient
  Method Without the Agonizing Pain'' by Jonathan Shewchuk.}.
In order to avoid getting lost in the weeds, it seems worthwhile to
start with a roadmap:
\begin{itemize}
\item
  We begin with the observation that if vectors $x^{(0)}, \ldots,
  x^{(k)}$ are increasingly good approximations to $x$, then some
  linear combination of these vectors may produce an even better
  approximation.  If the original sequence is produced by a stationary
  iteration, these vectors span a {\em Krylov subspace}.
\item
  One can generally show that a big enough Krylov subspace will
  contain a good approximation to $x$.  Alas, this does not tell us
  how to find which vector in the space is best (or even good)!
  Attempting to minimize the norm of the error is usually impossible,
  but it is possible to minimize the residual (leading to GMRES or
  MINRES), an energy function (CG), or some other error-related
  quantity.
\item
  The basic framework of a Krylov subspace plus a method of choosing
  approximations from the space allows us to describe some theoretical
  properties of several iterations without telling us why (or if) we
  can implement them efficiently and stably.  A key practical point is
  the computation of well-conditioned bases for the Krylov subspaces,
  e.g., using the {\em Lanczos} algorithm (symmetric case) or the
  {\em Arnoldi} algorithm (nonsymmetric case).
\end{itemize}

\section*{From Stationary Methods to Krylov Subspaces}

At the end of the last lecture, we tried to motivate the idea that
we can improve the convergence of a stationary method by replacing
the sequence of guesses
\[
  x^0, x^1, \ldots \rightarrow x
\]
with {\em linear combinations}
\[
  \tilde{x}^k = \sum_{j=1}^k \alpha_{kj} x^j.
\]
We could always choose $\alpha_{kj}$ to be one for $k = j$ and zero
otherwise, in which case we have the original stationary method;
but by choosing the coefficients more carefully, we might do better.

We've so far written stationary methods as
\[
  M x^{j+1} = N x^j + b.
\]
This is equivalent to
\[
  x^{j+1} = x^j + M^{-1} r^j, \quad r^j \equiv b-Ax^j,
\]
or
\[
  x^{j+1} = R x^j + M^{-1} b
\]
where $R = I-M^{-1} A$ is the iteration matrix we've seen
in our previous analysis.  If $x^0 = M^{-1} b$, this gives
\[
  x^j = \sum_{i=1}^j R^i M^{-1} b.
\]
If we look at this expression closely, we might notice that
the space spanned by the first $k$ iterates of the stationary
method is all vectors of the form
\[
  \sum_{i=0}^j c_i R^i M^{-1} b.
\]
If we look a little harder, we might observe that this is equivalent
to the space of all vectors of the form
\[
  \sum_{i=0}^j c_i (M^{-1} A)^i M^{-1} b = p(M^{-1} A) M^{-1} b
\]
where $p(z) = \sum_{i=1}^j c_i z^i$ is a polynomial of degree at most
$j$.

In general, the $d$-dimensional Krylov subspace generated by
a matrix $A$ and vector $B$ is
\begin{align*}
  \calK_d(A,b)
  &= \operatorname{span}\{ b, Ab, A^2 b, \ldots, A^{d-1} b \} \\
  &= \{ p(A) b : p \in \calP_{d-1} \}.
\end{align*}
As we have just observed, the iterates of a stationary method
form a basis for nested Krylov subspaces generated by $M^{-1}A$
and $M^{-1}b$.  If the stationary method converges, we know
the Krylov subspaces will eventually contain pretty good
approximations to $A^{-1} b$.  Let's now spell this out a little
more carefully.

\section*{The Power of Polynomials}

We showed a moment ago that the first $m$ iterates of a stationary
method form a basis for the space
\[
  \calK_{m+1}(R, M^{-1}b) = \calK_{m+1}(M^{-1} A, M^{-1} b)
\]
What can we say about the quality of this space for approximation?  As
it turns out, this turns into a question about polynomial
approximations.  We will not spell out all the details (nor will this
appear on any exams or homework problems for this class), but it's
worth spending a few moments giving an idea of what happens.

We have seen that the iterates of the stationary method are
\[
  x^{(k)} = x + e^{(k)} = x + R^k e^{(0)}
\]
We would like to take a linear combination
\[
  \tilde x^{(m)} = \sum_{k=0}^m \gamma_{mk} x^{(k)} = p_m(1) x + p_m(R) e^{(0)}
\]
where $p_m(z) = \sum_{k=0}^d \gamma_{mk} z^k$.  Moreover,
if $R$ is diagonalizable with $R = V \Lambda V^{-1}$, then
\[
  p_m(R) = V p(\Lambda) V^{-1}.
\]
For any $p_m$ with $p_m(1) = 1$, we have
\begin{align*}
\|\tilde{e}^{(m)}\|
&= \|\tilde{x}^{(m)}-x\| \\
&= \|p_m(R) e^{(0)}\| = \|V p(\Lambda) V^{-1} e^{(0)}\| \\
&\leq \kappa(V) \max_{\lambda_j} |p(\lambda_j)| \|e^{(0)}\|.
\end{align*}
Hence, we would really like to choose the polynomial that is one
at $1$ and as small as possible on each of the eigenvalues.

If all eigenvalues $\lambda_j$ of $R$ are real, then we have
\[
  \max_{\lambda_j} |p(\lambda_j)| \leq \max_{|z|<\rho(R)} |p(z)|,
\]
and a reasonable way to choose polynomials is to minimize $|p_m(z)|$
on $[-\rho(R),\rho(R)]$ subject to the constraint $p_m(1) = 1$.  The
solution to this problem is the {\em scaled Chebyshev polynomials},
with which we can show that the optimal $p_m$ satisfies
\begin{align*}
  p_m(z)
  & \leq \frac{2}{1+m\sqrt{2/(1-\rho(R))}} \\
  & = 2(1-m\sqrt{2(1-\rho(R))}) + O(m^2(1-\rho(R)).
\end{align*}
While the number of steps for the basic stationary iteration to
reduce the error by a fixed amount scales roughly like
$(1-\rho(R))^{-1}$, the number of steps to reduce the bound on the
optimal error scales like $(1-\rho(R))^{-1/2}$.

While the Chebyshev bounds are correct, and involve a beautiful bit of
approximation theory, they are limited.  For one thing, they fall
apart on non-normal matrices with complex spectra.  Even in the SPD
case, these bounds are often highly pessimistic in practice.  When the
eigenvalues of $R$ come in clusters, a relatively low degree
polynomial can do a good job of approximating $\lambda^{-1}$ at each
of the clusters, and hence a relatively low-dimensional Krylov
subspace may provide excellent solutions.

All of this is to say that the detailed convergence theory for Krylov
subspace methods can be quite complicated, but understanding a little
about the eigenvalues of the problem can provide at least a modicum of
insight.

For theoretical work, we are fine writing Krylov subspaces as
polynomials in $R$ applied to $M^{-1} b$.  In practical computations,
though, we need a basis.  Because the iterates of the stationary
method are converge to $x$, they tend to form a very ill-conditioned
basis.  We would like to keep the same Krylov subspace, but have a
different basis -- say, for instance, an orthonormal basis.  We turn
to this task next.

\section*{The Lanczos Idea}

What is good about the ``power basis'' for a Krylov subspace?  That is,
why might we like to write
\[
  \calK_m(A,b) = \operatorname{span}\{ b, Ab, A^2b, \ldots, A^{m-1}b \}
\]
rather than choosing a different basis?  Though it's terrible for
numerical stability, there are two features of the basis that are
attractive:
\begin{itemize}
\item
  The power bases span nested Krylov subspaces.  Given the vectors
  $b, \ldots, A^{m-1} b$ spanning $\calK_m(A,b)$, we only need one
  more vector ($A^d b$) to span $\calK_{m+1}(A,b)$.
\item
  Each successive vector can be computed from the previous vector
  with a single multiplication by $A$.  There is no other overhead.
\end{itemize}
While we dislike the power basis from the perspective of stability,
we would like to keep these attractive features for any alternative
basis.

We've already described one approach to converting the vectors in
a power basis into a more convenient set of vectors.  Define the matrix
\[
  X^{(m)} = \begin{bmatrix} b & Ab & A^2 b & \ldots & A^{m-1} b \end{bmatrix}
\]
and consider the economy QR decomposition
\[
X^{(m)} = Q^{(m)} R^{(m)}, \quad
Q^{(m)} = \begin{bmatrix} q_1 & q_2 & \ldots & q_{m} \end{bmatrix}.
\]
The columns of $Q^{(m)}$ are orthonormal and for any $k \leq m$, the
first $k$ columns of $Q^{(m)}$ span the same space as the first $k$
columns of $X^{(m)}$.  But forming $X^{(m)}$ and running QR is
unattractive for two reasons.  First, a dense QR decomposition may
involve a nontrivial amount of extra work, particularly as $m$ gets
large; and second, simply forming the rounded version of $X^{(m)}$ is
enough to get us into numerical trouble, even if we were able to run
QR with no additional rounding errors.  We need a better approach.

One helpful observation is that
\[
  \calR(AQ^{(k)}) = \calR(AX^{(k)}) \subseteq
  \calR(X^{(k+1)}) = \calR(Q^{(k+1)}).
  \]
That is, $Aq_k$ can always be written as a linear combination
of $q_1, \ldots, q_{k+1}$.  In matrix terms, this means we can write
\[
  AQ^{(k)} = Q^{(k+1)} \bar{H}^{(k)}
\]  
where
\[
\bar{H}^{(m)} =
\begin{bmatrix}
  h_{11} & h_{12} & h_{13} & \ldots & h_{1,k-1} & h_{1k} \\
  h_{21} & h_{22} & h_{23} &        & h_{2,k-1} & h_{2k} \\
  0     & h_{32} & h_{33} &        & h_{3,k-1} & h_{3k} \\
        & 0      & h_{43} &       & h_{4,k-1} & h_{4k} \\
        &        & \ddots & \ddots & \vdots & \vdots \\
        &        &        & 0      &  h_{k,k-1} & h_{kk} \\
        &        &        &        &  0        & h_{k+1,k}
\end{bmatrix}.
\]
A matrix with this structure (all elements below the first subdiagonal
equal to zero) is called {\em upper Hessenberg}.  Alternately,
we write
\[
  AQ^{(k)} = Q^{(k)} H^{(k)} + q_{k+1} h_{k+1,k}
\]
where $H^{(k)}$ is the square matrix consisting of all but the last
row of $\bar{H}^{(k)}$.  This formula is sometimes called an
{\em Arnoldi decomposition}; it turns out to be crucial in the
development of GMRES, one of the most popular iterative solvers for
nonsymmetric linear systems.  For the moment, though, we want to
focus on the symmetric case, and so we will move on.

When $A$ is symmetric, we have that
\[
  (Q^{(k)})^T A Q^{(k)} = H^{(k)}
\]
is also symmetric, as well as being upper Hessenberg; in this case,
the matrix $H^{(k)}$ is actually {\em tridiagonal}, and we write
$H^{(k)}$ as $T^{(k)}$ to emphasize this fact.  Conventionally,
$T^{(k)}$ is written as
\[
T^{(k)} =
\begin{bmatrix}
  \alpha_1 & \beta_1 \\
  \beta_1 & \alpha_2 & \beta_2 \\
          & \beta_2 & \alpha_3 & \ddots \\
          &         & \ddots & \ddots & \beta_{k-1} \\
          &         &        & \beta_{k-1} & \alpha_k
\end{bmatrix}.
\]
Converting from matrix notation back into vector notation, we have
\[
  A q_{k} = \beta_{k-1} q_{k-1} + \alpha_k q_k + \beta_k q_{k+1},
\]
which we can rearrange as
\[
  \beta_k q_{k+1} = A q_k - \alpha_k q_k - \beta_{k-1} q_{k-1}. 
\]
where
\begin{align*}
  \alpha_k &= q_k^T A q_k \\
  \beta_{k-1} &= q_k^T A q_{k-1} = q_{k-1}^T A q_k.
\end{align*}
Putting everything together, we have the {\em Lanczos iteration},
in which we obtain each successive vector $q_{k+1}$ by forming $Aq_k$,
orthogonalizing against the $q_k$ and $q_{k-1}$ by Gram-Schmidt,
and normalizing.

Presented as a {\em fait accompli}, the Lanczos iteration looks like
magic.  It seems even more like magic when one realizes that despite
an instability in the iteration (because of the use of Gram-Schmidt
for orthonormalization at each step), the iteration still produces
useful information in floating point.  The Lanczos iteration is the
basis for one of the most popular iterative methods for solving
eigenvalue problems, and in that setting it is important to
acknowledge and deal with the instability in the method.  For the
moment, though, we are still interested in solving linear systems,
and the method of Conjugate Gradients (also built on Lanczos)
turns out to still work great.

\section{Addendum: Three-Term Recurrences}

The Lanczos iteration allows us to generate a sequence of orthonormal
vectors using a three-term recurrence.  As it turns out, the same
approach leads to three-term recurrences that generate families of
orthogonal polynomials, including the Chebyshev polynomials mentioned
in passing early in the lecture and the Legendre polynomials that play
a significant role in the development of Gaussian quadrature.  I
consider the details beyond of these connections to be beyond the
scope of the current class.  But the connections are too beautiful and
numerous to not mention that they exist.  I would hate for you to walk
away from this class with the impression that the mathematical
development of the Lanczos iteration is only some quirky trick in
numerical linear algebra that gets you part of the way to CG.

\end{document}
