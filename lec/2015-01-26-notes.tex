\documentclass[12pt, leqno]{article}
\usepackage{amsfonts}
\usepackage{amsmath}
\usepackage{fancyhdr}
\usepackage{hyperref}
\usepackage{tikz}
\usepackage{pgfplots}
\usepackage{listings}

\newcommand{\bbR}{\mathbb{R}}
\newcommand{\bbC}{\mathbb{C}}

\newcommand{\hdr}[2]{
  \pagestyle{fancy}
  \lhead{Bindel, Spring 2015}
  \rhead{Numerical Analysis (CS 4220)}
  \fancyfoot{}
  \begin{center}
    {\large{\bf #1}} \\
    Due: #2
  \end{center}
  \lstset{language=matlab,columns=flexible}
}

\newcommand{\phdr}[1]{
  \pagestyle{fancy}
  \lhead{Bindel, Spring 2015}
  \rhead{Numerical Analysis (CS 4220)}
  \fancyfoot{}
  \begin{center}
    {\large{\bf #1}}
  \end{center}
  \lstset{language=matlab,columns=flexible}
}


\begin{document}
\hdr{2015-01-26}

% CV 12 commandments
% ==================
% Matvec = linear combo of columns
% Inner product = sum of products
% Order of ops is important
% Matrix * diag = col scaling
% Diag * matrix = row scaling
% Never form explicit diag matrix
% Never form explicit rank 1
% Matrix * matrix = collection of matrix * vector
% Matrix * matrix = dot products
% Matrix * matrix = sum of rank one
% Matrix * matrix = linear combo of rows from second matrix
% Matrix * matrix = linear combo of cols from first matrix

% DSB
% ===
% Blocking
% Dynamic programming
% Sparsity and implicit multiply

\section*{Matrix algebra versus linear algebra}

\begin{enumerate}
\item
  Matrices are extremely useful.  So are linear transformations.  But note
  that matrices and linear transformations are {\em different} things!
  Matrices {\em represent} finite-dimensional linear transformations with
  respect to particular bases.  Change the bases, and you change the
  matrix, if not the underlying operator.  Much of the class will be about
  finding the right basis to make some property of the underlying transformation
  obvious, and about finding changes of basis that are ``nice'' for
  numerical work.

\item
  A linear transformation may correspond to different matrices depending on the
  choice of basis, but that doesn't mean the linear transformation is always the
  thing.  For some applications, the matrix itself has meaning, and the
  associated linear operator is secondary.  For example, if I look at 
  an adjacency matrix for a graph, I probably really do care about the matrix --
  not just the linear transformation.

\item
  Sometimes, we can apply a linear transformation even when we don't have an
  explicit matrix.  For example, suppose $F : \bbR^n \rightarrow
  \bbR^m$, and I want to compute 
  $\partial F / \partial v|_{x_0} = (\nabla F(x_0)) \cdot v$.
  Even without an explicit matrix for $\nabla F$, I can compute
  $\partial F / \partial v|_{x_0} \approx F(x_0 + hv)-F(x_0))/h$.
  There are many other linear transformations, too, for which it is
  more convenient to apply the transformations than to write down the
  matrix -- using the FFT for the Fourier transform operator, for
  example, or fast multipole methods for relating charges to potentials
  in an $n$-body electrostatic interaction.

\end{enumerate}


\section*{Matrix-vector multiply}

Let us start with a very simple \matlab\ program for matrix-vector
multiplication:
\begin{verbatim}
  function y = matvec1(A,x)
  % Form y = A*x (version 1)

  [m,n] = size(A);
  y = zeros(m,1);
  for i = 1:m
    for j = 1:n
      y(i) = y(i) + A(i,j)*x(j);
    end
  end
\end{verbatim}
We could just as well have switched the order of the $i$ and $j$ loops
to give us a column-oriented rather than row-oriented version of the algorithm.
Let's consider these two variants, written more compactly:
\begin{verbatim}
  function y = matvec2_row(A,x)
  % Form y = A*x (row-oriented)

  [m,n] = size(A);
  y = zeros(m,1);
  for i = 1:m
    y(i) = A(i,:)*x;
  end


  function y = matvec2_col(A,x)
  % Form y = A*x (column-oriented)

  [m,n] = size(A);
  y = zeros(m,1);
  for j = 1:n
    y = y + A(:,j)*x(j);
  end
\end{verbatim}

It's not too surprising that the builtin matrix-vector multiply routine in
\matlab\ runs faster than either of our {\tt matvec2} variants, but there
are some other surprises lurking.  Try timing each of these matrix-vector
multiply methods for random square matrices of size 4095, 4096, and 4097,
and see what happens.  Note that you will want to run each code many times
so that you don't get lots of measurement noise from finite timer granularity;
for example, try
\begin{verbatim}
  tic;          % Start timer
  for i = 1:100 % Do enough trials that it takes some time
    % ...         Run experiment here
  end
  toc           % Stop timer
\end{verbatim}

% Matrix multiplication and blocking
% Memory access and vectorization issues; BLAS routines
% Matrix representation

\section*{Basic matrix-matrix multiply}

The classic algorithm to compute $C := C + AB$ is
\begin{verbatim}
  for i = 1:m
    for j = 1:n
      for k = 1:p
        C(i,j) = C(i,j) + A(i,k)*B(k,j);
      end
    end
  end
\end{verbatim}
This is sometimes called an {\em inner product} variant of
the algorithm, because the innermost loop is computing a dot
product between a row of $A$ and a column of $B$.  We can
express this concisely in MATLAB as
\begin{verbatim}
  for i = 1:m
    for j = 1:n
      C(i,j) = C(i,j) + A(i,:)*B(:,j);
    end
  end
\end{verbatim}
There are also {\em outer product} variants of the algorithm
that put the loop over the index $k$ on the outside, and thus
computing $C$ in terms of a sum of outer products:
\begin{verbatim}
  for k = 1:p
    C = C + A(:,k)*B(k,:);
  end
\end{verbatim}

\section*{Blocking and performance}

The basic matrix multiply outlined in the previous section will
usually be at least an order of magnitude slower than a well-tuned
matrix multiplication routine.  There are several reasons for this
lack of performance, but one of the most important is that the basic
algorithm makes poor use of the {\em cache}.
Modern chips can perform floating point arithmetic operations much
more quickly than they can fetch data from memory; and the way that
the basic algorithm is organized, we spend most of our time reading
from memory rather than actually doing useful computations.
Caches are organized to take advantage of {\em spatial locality},
or use of adjacent memory locations in a short period of program execution;
and {\em temporal locality}, or re-use of the same memory location in a
short period of program execution.  The basic matrix multiply organizations
don't do well with either of these.
A better organization would let us move some data into the cache
and then do a lot of arithmetic with that data.  The key idea behind
this better organization is {\em blocking}.

When we looked at the inner product and outer product organizations
in the previous sections, we really were thinking about partitioning
$A$ and $B$ into rows and columns, respectively.  For the inner product
algorithm, we wrote $A$ in terms of rows and $B$ in terms of columns
\[
  \begin{bmatrix} a_{1,:} \\ a_{2,:} \\ \vdots \\ a_{m,:} \end{bmatrix}
  \begin{bmatrix} b_{:,1} & b_{:,2} & \cdots & b_{:,n} \end{bmatrix},
\]
and for the outer product algorithm, we wrote $A$ in terms of colums
and $B$ in terms of rows
\[
  \begin{bmatrix} a_{:,1} & a_{:,2} & \cdots & a_{:,p} \end{bmatrix}
  \begin{bmatrix} b_{1,:} \\ b_{2,:} \\ \vdots \\ b_{p,:} \end{bmatrix}.
\]
More generally, though, we can think of writing $A$ and $B$ as 
{\em block matrices}: 
\begin{align*}
  A &=
  \begin{bmatrix}
    A_{11} & A_{12} & \ldots & A_{1,p_b} \\
    A_{21} & A_{22} & \ldots & A_{2,p_b} \\
    \vdots & \vdots &       & \vdots \\
    A_{m_b,1} & A_{m_b,2} & \ldots & A_{m_b,p_b}
  \end{bmatrix} \\
  B &=
  \begin{bmatrix}
    B_{11} & B_{12} & \ldots & B_{1,p_b} \\
    B_{21} & B_{22} & \ldots & B_{2,p_b} \\
    \vdots & \vdots &       & \vdots \\
    B_{p_b,1} & B_{p_b,2} & \ldots & B_{p_b,n_b}
  \end{bmatrix} 
\end{align*}
where the matrices $A_{ij}$ and $B_{jk}$ are compatible for matrix
multiplication.  Then we we can write the submatrices of $C$ in terms
of the submatrices of $A$ and $B$
\[
  C_{ij} = \sum_k A_{ij} B_{jk}.
\]

\section*{The lazy man's approach to performance}

An algorithm like matrix multiplication seems simple, but there is a
lot under the hood of a tuned implementation, much of which has to do
with the organization of memory.  We often get the best ``bang for our
buck'' by taking the time to formulate our algorithms in block terms,
so that we can spend most of our computation inside someone else's
well-tuned matrix multiply routine (or something similar).  There are
several implementations of the Basic Linear Algebra Subroutines
(BLAS), including some implementations provided by hardware vendors
and some automatically generated by tools like ATLAS.  The best BLAS
library varies from platform to platform, but by using a good BLAS
library and writing routines that spend a lot of time in {\em level 3}
BLAS operations (operations that perform $O(n^3)$ computation on
$O(n^2)$ data and can thus potentially get good cache re-use), we can
hope to build linear algebra codes that get good performance across
many platforms.

This is also a good reason to use \matlab: it uses pretty good BLAS libraries,
and so you can often get surprisingly good performance from it for the types
of linear algebraic computations we will pursue.

\section*{Problems to ponder}

Unless otherwise stated, assume $A, B \in \bbR^{n \times n}$
(square real $n \times n$ matrices), $u, v, x, y$ are vectors in
$\bbR^n$, and $D = \operatorname{diag}(d)$ is a diagonal $n \times n$.
\begin{enumerate}
\item
  Describe the effect of pre- and post-multiplying $A$ by $D$;
  that is, what are $DA$ and $AD$?
\item
  How many floating point operations are needed to evaluate the
  following (assuming ordinary order of operations)?
  \begin{enumerate}
  \item $(uv^T) A$
  \item $u (v^T A)$
  \item $A (u v^T) B$
  \item $(A u) (v^T V)$
  \item $A D x$
  \item $A (Dx)$
  \end{enumerate}
\item
  Describe a brief snippet of MATLAB code to form the most efficient
  versions of the above expressions.
\item
  The standard tridiagonal matrix $T_N \in \bbR^{N \times N}$ acts
  on the vector $u$ in the following way:
  \[
    (Tu)_i = -u_{i-1} + 2u_i - u_{i+1}
  \]
  with the convention $u_0 = u_{N+1} = 0$.
  \begin{enumerate}
  \item What is $T_5$, written explicitly?
  \item Write a MATLAB snippet to evaluate $Tu$ in $O(N)$ time.
  \end{enumerate}
\item
  Let $E \in \bbR^{n \times n}$ be the matrix of all ones.
  Describe an $O(n)$ approach to compute $Ev$.
\item
  The operation $\operatorname{triu}(E)$ takes the upper triangular
  part of $E$; for example, for $n = 3$, we have
  \[
  \operatorname{triu}(E) =
  \begin{bmatrix}
    1 & 1 & 1 \\
    0 & 1 & 1 \\
    0 & 0 & 1 
  \end{bmatrix}
  \]
  In general, describe an $O(n)$ approach to compute
  $\operatorname{triu}(E) v$.
\end{enumerate}

\end{document}
